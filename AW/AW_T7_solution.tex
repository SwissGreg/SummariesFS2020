\documentclass[8pt]{extreport}
\usepackage{parskip}
\usepackage{amsmath}
\usepackage{amssymb}
\usepackage{fancyhdr}
\usepackage{enumitem}
\usepackage{geometry}
\geometry{a4paper, margin = 1in}
\setlength{\headheight}{15.2pt}
\pagestyle{fancy}
\fancyhead[L]{A$\&$W T7}
\fancyhead[R]{Matteo Nussbaumer, Gregory Rozanski}
\title{Analysis Serie 4}
\begin{document}
\paragraph{\underline{Exercise 1- Second Moment Method:}}
\begin{enumerate}[label = (\alph*)]
\item Wenn n ungerade ist, ist es unmöglich n farbenwechseln zu bekommen
$\Rightarrow f_X(n) = 0$\\
Für die binomialverteilung gilt: $f_X(x) = \binom{n}{x}p^x(1-p)^{n-1}, \quad x \in \{0,1,...,n\}$\\
$\Rightarrow f_X(n) = \frac{1}{2}^n \neq 0$\\
$\Rightarrow$ X ist nicht binomialverteilt
\item Jede Perle ist rot mit Wahrscheinlichkeit $\frac{1}{2}$ und sonst blau unabhängig von den anderen Perlen.\\
$\Rightarrow$ Die Wahrscheinlichkeit dass perle i eine andere Farbe von perle i+1 hat ist $\frac{1}{2}$\\
$\Rightarrow \mathbb{E}[X_i] = \frac{1}{2}$\\
$\Rightarrow \mathbb{E} [X] = \mathbb{E}[X_1] + \mathbb{E}[X_2] + ... + \mathbb{E}[X_n]$\\
$\Rightarrow \mathbb{E} [X] = \frac{n}{2}$ 
\item $Pr[X \geq \mathbb{E}[X] + t]$\\
$\Rightarrow \mathbb{E}[X]  = \displaystyle\sum_{x \in W_X} x \cdot Pr[X=x]$\\
$ = \displaystyle\sum_{x < \mathbb{E}[X] + t} x \cdot Pr[X=x] + \displaystyle\sum_{ x \geq \mathbb{E}[X] + t} x \cdot Pr[X=x]$\\
$\Rightarrow \mathbb{E}[X] \geq  \displaystyle\sum_{ x \geq \mathbb{E}[X] + t} x \cdot Pr[X=x] \geq \displaystyle\sum_{ x \geq \mathbb{E}[X] + t} (\mathbb{E}[X]+t) \cdot Pr[X=x]  = (\mathbb{E}[X] + t) \cdot \displaystyle\sum_{ x \geq \mathbb{E}[X] + t} Pr[X=x]$\\
$\Rightarrow Pr[X \geq \mathbb{E}[X] + t] \leq \frac{\mathbb{E}[X]}{(\mathbb{E}[X] + t)}$ (Ungleichung von Markov)
\item $\mathbb{E}[X_i \cdot X_j]  =$\\
\begin{itemize}
\item falls i $\neq$ j $\rightarrow \mathbb{E}[X_i \cdot X_j] = \mathbb{E}[X_i] \cdot \mathbb{E}[X_j] = \frac{1}{4}$
\item falls i = j $\rightarrow \mathbb{E}[X_i \cdot X_j] = \mathbb{E}[X_i \cdot X_i] = \frac{1}{2}$
\end{itemize}
$\Rightarrow \mathbb{E}[X^2] = \displaystyle\sum_{i=1}^{n} \displaystyle\sum_{j=1}^{n} \mathbb{E}[X_i \cdot X_j] = \frac{n}{2} + \frac{n^2 -n}{4} = \frac{n^2 + n}{4}$\\
$\Rightarrow Var[X] = \frac{n^2 + n }{4} - \frac{n^2}{4} = \frac{n}{4}$
\item Die Chebyshev Ungleichung besagt: $Pr[|X - \mathbb{E}[X]| \geq t] \leq \frac{Var[X]}{t^2}$\\
$Pr[|X - \mathbb{E}[X]| \geq t]  = Pr[X \leq \mathbb{E}[X] -t] + Pr[X \geq \mathbb{E}[X] + t]$\\
$\Rightarrow Pr[|X - \mathbb{E}[X]| \geq t] \geq Pr[X \geq \mathbb{E}[X] + t]$\\
$\Rightarrow  Pr[X \geq \mathbb{E}[X] + t] \leq \frac{Var[X]}{t^2}$\\
\item
\begin{itemize}
\item $P_{Mark} = \frac{\mathbb{E}[X]}{(\mathbb{E}[X] + t)} = \frac{\frac{n}{2}}{\frac{n}{2} +t} \leq \frac{1}{4}$\\
$\Rightarrow \frac{n}{2}  \leq (\frac{n}{2} + t) \cdot \frac{1}{4}$\\
$\Rightarrow \frac{n}{2} \leq \frac{n}{8} + \frac{t}{4}$\\
$\Rightarrow \frac{3n}{8} \leq \frac{t}{4}$\\
$\Rightarrow \frac{3n}{2} \leq t$\\
\item $P_{Cheb} = \frac{Var[X]}{t^2}  = \frac{n}{4t^2} \leq \frac{1}{4}$\\
$\Rightarrow n \leq \frac{4t^2}{4} = t^2$\\
$\Rightarrow \sqrt{n} \leq t$
\end{itemize}
\end{enumerate}
\end{document}