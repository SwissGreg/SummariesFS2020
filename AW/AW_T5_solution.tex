\documentclass{report}
\usepackage{parskip}
\usepackage{amsmath}
\usepackage{amssymb}
\usepackage{fancyhdr}
\usepackage{enumitem}
\setlength{\headheight}{15.2pt}
\pagestyle{fancy}
\fancyhead[L]{A$\&$W T5}
\fancyhead[R]{Matteo Nussbaumer, Gregory Rozanski}
\title{Analysis Serie 4}
\begin{document}
\paragraph{Exercise 1}
\begin{enumerate}[label = \alph*)]
\item Wir definieren die Mengen:
\begin{center}
$\Omega = \{ (i_R,j_B) | i,j \in \{,1,2,3,4,5,6\}\}\quad \rightarrow \quad  |\Omega| = 36$\\
$A = \{(2_R,j_B),(4_R,j_B),(6_R,j_B)|j \in \{,1,2,3,4,5,6\}\} \quad \rightarrow \quad |A| = 18$\\
$B = \{(i_R,5_B) | i \in \{1,2,3,4,5,6\}\}\quad \rightarrow \quad |B| = 6$\\
$C = \{(1_R,2_B),(1_R,3_B),(1_R,5_B),(2_R,1_B),(3_R,1_B),(5_R,1_B)\}$ \\
$\rightarrow |C| = 6$\\
$D =  \{(1_R,2_B),(1_R,5_B),(2_R,1_B),(2_R,4_B),(3_R,3_B),(4_R,2_B),(5_R,1_B), (3_R,6_B), (6_R,3_B),(6_R,6_B)\}$ \\
$\rightarrow |D| = 10$\\
\end{center}
$\Rightarrow \quad $$Pr[A] = \frac{1}{2}$,
$Pr[B] = \frac{1}{6}$,
$Pr[C] = \frac{1}{6}$,
$Pr[D] = \frac{5}{18}$
\begin{center}
$Pr[A \cap C] = \frac{1}{36}$\\
$Pr[A \cap B] = \frac{3}{36} = \frac{1}{12}$\\
$Pr[B \cap C] = \frac{1}{36}$\\
$Pr[C \cap D] = \frac{4}{36} = \frac{1}{9}$
\end{center}
$\Rightarrow \quad Pr[A|C] = \frac{Pr[A \cap C]}{Pr[C]} = \frac{1}{6}$\\
Zwei beliebige Ereignisse A,B sind unabhängig wenn gilt:
\begin{center}
 $Pr[A \cap B] = Pr[A] \cdot Pr[B]$
\end{center}
$\Rightarrow Pr[A \cap B] = \frac{1}{12} = \frac{1}{2} \cdot \frac{1}{6} \quad \Rightarrow \quad$ A und B sind unabhängig.\\
$\Rightarrow Pr[B \cap C] = \frac{1}{36} = \frac{1}{6} \cdot {1}{6} \quad \Rightarrow \quad$ B und C sind unabhängig.\\
$\Rightarrow Pr[C \cap D] = \frac{1}{9} \neq \frac{1}{6} \cdot \frac{5}{18} \quad \Rightarrow \quad $C und D sind abhängig.
\item Es gibt 3 Hot Dogs (HD), eine ist auf eine Seite gebissen, der zweite auf beide Seiten gebissen und der dritte ist unverzehrt. Zusätzlich wissen wir dass wir ein HD bekommen dass auf eine Seite nicht gebissen ist.\\
$\Rightarrow$ Wir haben entweder der HD der auf eine Seite gebissen ist oder der HD der gar nicht gebissen ist, wir bezeichnen sie  HD1 und HD2\\
$\Rightarrow$ Wir nehmen an dass HD1 auf der Linke Seite gebissen ist.\\
$\Rightarrow \Omega=\{ HD1_L,HD2_L, HD2_R\}$\\
(Entweder sehen wir den HD1 oder einer von den ungebissenen enden HD2)\\
$\Rightarrow \quad $ A:= " Wir haben HD2 "\\
$\Rightarrow \quad Pr[A] = \frac{2}{3}$


\item Zuerst berechnen wir die Wahrscheinlichkeit, dass David die folge ZKKZ in 4 Würfe bekommt.\\
$\Rightarrow \quad Pr[E_i] = \frac{1}{2^4} = \frac{1}{16}$\\
Insgesamt macht David 10 Würfe d.h die Folge ZKKZ kann maximal 3 mal vorkommen.
$\Rightarrow \quad$ Alle Terme der Siebformel mit welche die Schnittmenge 4 oder mehr Ereignisse sind 0 und wir kriegen die folgende Gleichung:\\
\begin{align*}
\Rightarrow  \quad Pr[E_1 \cup \dots \cup E_7] =& \displaystyle\sum_{i=1}^{7} Pr[E_i] \\ &- Pr[E_1 \cap E_4] - Pr[E_1 \cap E_5]  - Pr[E_1 \cap E_6] - Pr[E_1 \cap E_7]\\& - Pr[E_2 \cap E_5] - Pr[E_2 \cap E_6]  - Pr[E_2 \cap E_7] \\& - Pr[E_3 \cap E_6]- Pr[E_3 \cap E_7] -Pr[E_4 \cap E_7]+ Pr[E_1 \cap E_4 \cap E_7] 
\end{align*}
$\Rightarrow \quad Pr[E_1 \cup \dots \cup E_7] = \frac{7}{16} - \frac{6}{16^2} - \frac{4}{2^7} + \frac{1}{2^{10}} = 0.3837890625$
 























\end{enumerate}
\end{document}