\documentclass[8pt]{extreport}
\usepackage{parskip}
\usepackage{amsmath}
\usepackage{amssymb}
\usepackage{fancyhdr}
\usepackage{graphicx}
\usepackage{enumitem}
\usepackage{mathtools}
\usepackage{geometry}
\geometry{a4paper, margin=1in}
\setlength{\headheight}{15.2pt}
\pagestyle{fancy}
\fancyhead[L]{Analysis Serie 6}
\fancyhead[R]{Gregory Rozanski}
\title{Analysis Serie 6}
\begin{document}
\begin{enumerate}[label = \textbf{6.\arabic*}]
\item \textbf{Reihen reellen Zahlen}
\begin{enumerate}[label = (\alph*)]
\item $\rho = \lim\limits_{n \to \infty}\big|\frac{n!}{(n+1)!}\big| = \lim\limits_{n \to \infty}\big|\frac{1}{n+1}\big| = 0 \Rightarrow x \in  (-\infty, 0 ]$
\item $\displaystyle\sum_{n=1}^{\infty}2^n e^{2nx} = \displaystyle\sum_{n=1}^{\infty}(2e^{2x})^n \rightarrow $ Die geometrische Reihe konvergiert für $|q| < 1$ \\
$\Rightarrow 2e^{2x} < 1$\\
$\Rightarrow e^{2x} < \frac{1}{2}$\\
$\Rightarrow 2x < ln\big(\frac{1}{2}\big)$\\
$\Rightarrow x < \frac{1}{2}ln\big(\frac{1}{2}\big) = ln\big(\frac{1}{\sqrt{2}}\big)$
$\Rightarrow$ Die Reihe konvergiert für $ x \in (-\infty, ln\big(\frac{1}{\sqrt{2}}\big) )$
\item  Wir benutzen den WK um den konvergenz radius zu berechnen.\\
$\Rightarrow \sqrt[n]{|a_n|} = \bigg|\frac{x^n}{(1-x)^{2n}}\bigg|^\frac{1}{n} = \frac{|x|}{|(1-x)^2|} = \frac{|x|}{(1-x)^2} = \frac{|x|}{(x^2 -2x +1)} < 1$\\
$\Rightarrow x < x^2-2x+1 \rightarrow 0 < x^2 - 3x + 1 \rightarrow x =\frac{3 \pm \sqrt{5}}{2} $\\
$\Rightarrow$ für $x \in (-\infty,\frac{3 - \sqrt{5}}{2}) \cup (\frac{3 + \sqrt{5}}{2}, \infty)$ konvergiert die Reihe
\end{enumerate}
\item \textbf{MC Fragen}
\begin{enumerate}[label = (\alph*)]
\item unstetig, wenn man den umgebung von 2 betrachtet gibt es ein sprung da $2 <x <3$ zu 3 aufgerundet wird
\item erste und zweite bedingung
\end{enumerate}
\item \textbf{Stetigkeit I (Please Correct)}
$f(x) = \frac{1}{x^2 + 4}$ finde ein $\rho$ für jedes $\epsilon$ sodass $|x -y| < \rho \Rightarrow |f(x) -f(y)| < \epsilon$:\\
$|f(x) -f(y)| = \big|\frac{1}{x^2 + 4}- \frac{1}{y^2+4}\big| = \frac{|y^2 + 4 - x^2 -4|}{|(x^2 + 4)(y^2+4)|} = \frac{|y^2 -x^2|}{|(x^2 + 4)(y^2+4)|} = \frac{|x+y| |x-y|}{|(x^2 + 4)(y^2+4)|}$\\ $\leq \frac{|x+y|}{|(x^2 + 4)(y^2+4)|}\rho = \frac{|x-y+y+y|}{|(x^2 + 4)(y^2+4)|}\rho \leq  \frac{|x-y| + |2y|}{|(x^2 + 4)(y^2+4)|}\rho \leq \frac{(\rho + |2y|)}{|(x^2 + 4)(y^2+4)|}\rho \leq (\rho + |2y|)\rho$ \\
$\Rightarrow |f(x) -f(y)| < (\rho + 2|y|)\rho$\\
$\Rightarrow \rho^2 - 2|y|\rho = \epsilon$\\
$\Rightarrow \rho^2 -2|y|\rho -\epsilon = 0$\\
$\Rightarrow \rho = \frac{2|y| \pm \sqrt{4y^2 + 4\epsilon}}{2} = |y|\sqrt{y^2 + \epsilon}$
\item \textbf{Zwischenwertsatz}
\begin{enumerate}[label = (\alph*)]
\item Wir betrachten eine zweite stetige funktion g(x) = f(x)-x mit g(0) = 0 und g(1) = f(1)-1\\
$\Rightarrow g(0) \geq 0$ und $g(1) \leq 0$\\
$\Rightarrow$ Aus dem Zwischenwertsatz folgt dass es ein $x \in [0,1]$ mit $g(x) = 0$\\
$\Rightarrow g(x) = f(x) - x = 0 \rightarrow f(x) = x$\\
$\Rightarrow$ f hat ein Fixpunkt
\item Wir betrachten eine zweite stetige funktion $g(x) = f(x) -f(x + \frac{1}{n})$\\
$\Rightarrow g(0) = f(0) - f(0 + \frac{1}{n}) = -f(\frac{1}{n})$\\
$\Rightarrow g(\frac{n-1}{n}) = f(\frac{n-1}{n}) - f(\frac{n-1}{n}) +f(\frac{1}{n}) = f(\frac{n-1}{n}) - f(\frac{n}{n}) = f(\frac{n-1}{n})$
\end{enumerate}
\end{enumerate}
\end{document}