\documentclass{report}
\usepackage{parskip}
\usepackage{amsmath}
\usepackage{amssymb}
\usepackage{fancyhdr}
\usepackage{enumitem}
\setlength{\headheight}{15.2pt}
\pagestyle{fancy}
\fancyhead[L]{Analysis Serie 4}
\fancyhead[R]{Gregory Rozanski}
\title{Analysis Serie 4}
\begin{document}
\paragraph{4.1 MC Fragen: Folgenkonvergenz}
\begin{enumerate}[label = \alph*)]
\item $\displaystyle\sum_{k=1}^{\infty} a_k$
\begin{itemize}
\item Wahr  $\rightarrow$ laut der Cauchy Kriterium ist die reihe genau dann konvergent falls $\forall \epsilon > 0 \ \ \exists N \geq 1 \text{ mit } \bigg|\displaystyle\sum_{k=n}^{m} a_k \bigg| < \epsilon \quad \forall m \geq n \geq N$
\item Wahr (begründung gleich wie oben)
\item Falsch $a_k := 1 \Rightarrow \displaystyle\sum_{k=1}^{\infty} sin(1) \text{ konvergiert absolut, aber } \displaystyle\sum_{k=1}^{\infty} 1$  divergiert
\item Falsch
\end{itemize}
\item $\displaystyle\sum_{k=1}^{\infty} a_k$ eine reelle Reihe mit $\forall k \in \mathbb{N} : a_k \leq 0$
\begin{itemize}
\item Wahr. Bew: $S_{n+1} - S_{n} = \displaystyle\sum_{k=1}^{n+1}a_k - \displaystyle\sum_{k=1}^{n}a_k = a_{n+1} \leq 0 \Rightarrow (S_{n})$ ist monoton fallend. Falls $(S_{n})$ nach unten beschränkt ist, folgt aus Weierstrass dass die Folge konvergiert.
\item Falsch. Bew sei $(a_k) := -\frac{1}{k}$\\
$\Rightarrow \quad (a_k)$ ist monoton wachsend und $\lim\limits_{n \to \infty} a_k = 0$\\
$\Rightarrow \quad$ Der Harmonische Reihe divergiert\\
$\Rightarrow \quad  \displaystyle\sum_{k=1}^{\infty}-\frac{1}{k}$ divergiert
\item Falsch. Bew: Gegenbeispiel negative Geometrische Reihe
\end{itemize}
\item $\phi: \mathbb{N^*} \rightarrow \mathbb{N^*}$ eine Abbildung, $\displaystyle\sum_{n=1}^{\infty}a_n$ eine Reihe und 
\begin{itemize}
\item Wahr. Bew: Da Definitionsmenge und Zielmenge gleich sind, ist es eine umordnung von $a_n$ daraus folgt die Summe der $b_n$'s konvergiert 
\item Wahr. Bew: Da $\phi$ injektiv ist, gilt $\phi(a) \neq \phi(b) | a\neq b$. Da die summe nach unendlich geht ist $\phi$ eine Umordnung der Summe der $b_n$'s konvergiert
\item Wahr. Bew: Satz von Dirichlet
\item Wahr. Bew: Satz von Dirichlet
\end{itemize}
\end{enumerate}
\paragraph{test}
\end{document}