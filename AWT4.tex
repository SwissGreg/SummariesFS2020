\documentclass{report}
\usepackage{parskip}
\usepackage{amsmath}
\usepackage{amssymb}
\usepackage{fancyhdr}
\usepackage{enumitem}
\setlength{\headheight}{15.2pt}
\pagestyle{fancy}
\fancyhead[L]{A$\&$W  T4}
\fancyhead[R]{Gregory Rozanski, Gion Vezzini}

\begin{document}
\paragraph{Exercise 1}
\begin{enumerate}[label = \alph*)]
\item Gegeben: k-Gerichte,  2k-Köche
$\Rightarrow$ wir bilden einen bipartiten Graph $G=(A\uplus B,E)$ bei dem eine Knotenpartition aus den 2k Kochen besteht und die andere aus den Gerichten wobei für jedes Gericht zwei knoten darstellt. Es gibt eine Kante {u,v} von der Koch u aus A nach Gericht v aus B genau dann, wenn der Koch weiss wie man das Gericht anrichtet. (Da jedes Gericht zwei Knoten hat werden immer zwei kanten eingefügt). Ein perfektes Matching entsteht, wenn zwei Köche aus A ein Gericht aus B kochen können.
\item Annahme: Es ist möglich alle Gerichte zu kochen\\
$\Rightarrow$ Es gibt ein perfektes Matching in unserem Graph nach konstruktion in a).\\
$\Rightarrow$ Es gibt genau dann ein Matching in einen Graphen $G=(A \uplus B,E)$ für dass gilt $|M|=|A|$ wenn:
\begin{center}
$|N(X)| \geq |X|$
\end{center}
$\Rightarrow$ Da es ein Matching der grösse $|M| = |A| =|B|$ gibt, und dies alle Gerichte beinhaltet, gilt es auch für alle Teilmengen von A oder B. Dies folgt aus dem Satz von Hall ("für alle $X\subseteq A$"). Nehmen wir eine Teilmenge X der k Gerichte. Da jeder Gericht 2 Köche zugeordnet wir wissen wir dass es $2\cdot|X|$ köche gibt welche gerichte aus X kochen können.
\item Analog zu b), dies folgt aus dem Satz von Hall.
\end{enumerate}
\paragraph{Exercise 2}
\begin{enumerate}[label = \alph*)]
\item $\mathit{x}(G) =4$\\
Wir geben den Mittelpunkt der Kreis die Farbe 1.\\
$\Rightarrow$ Da dieses Knoten mit alle andere Knoten verbunden ist müssen die knoten am äusseren Kreis eine farbe $\neq$ 1 haben.\\
$\Rightarrow$ Kreise mit ungeraden länge haben einen chromatische zahl von 3.\\
$\Rightarrow$ Insgesamt braucht man 4 farben um den Graph zu färben.
\item Falsch, betrachte ein Baum mit 4 knoten. Färbt man die beide blätter zuerst, kriegen sie die Farbe 1, danach färbt man den Wurzel mit 2 (da es mit eine der Blätter verbunden ist). Das letzte Knoten is verbunden mit der Wurzel und ein Blatt d.h die Farben 1,2 können nicht verwendet werden.\\
$\Rightarrow$ Es gibt einen Reihenfolge von Knoten wo der Greedy-Algorithmus $>2$ farben braucht
\item
\begin{itemize}
\item $\mathit{x}(G) = 2\quad \rightarrow \quad \langle v_1,v_2,v_3,v_4,v_5,v_6 \rangle$
\item $\mathit{x}(G) = 3\quad \rightarrow \quad \langle v_1,v_4,v_2,v_3,v_5,v_6 \rangle$
\end{itemize}
\item Gegeben ein Graph G mit chromatische zahl $\mathit{x}(G) = c $\\
$\Rightarrow$ nach definition der chromatische zahl kann man G mit c farben färben\\
$\Rightarrow$ G ist c-partit\\
$\Rightarrow$ Der Greedy-Algorithmus kann alle Knoten der erste Partition die farbe 1 geben, danach der 2te Partition die farbe 2 usw. die c-te Partition kriegt die farbe c.\\
$\Rightarrow$ Es existiert für jedes Graph G ein knoten reihenfolge sodass der Greedy-Algorithmus $\mathit{x}(G)$ farben braucht.\\
\end{enumerate}
\paragraph{Exercise 3}
\begin{enumerate}[label = \alph*)]
\item
\begin{enumerate}[label = \arabic*.]
\item Wir betrachten einen fairen Münze (d.h Kopf und Zahl sind gleich wahrscheinlich), welche zwei mal geworfen wird:
\begin{center}
$\Omega = \{KK,KZ,ZK,ZZ\}$\\
A:= Man wirft zwei mal K\\
B:= Man wirft ein mal Z\\
$A \cup B$:= Man wirft zwei mal Kopf oder ein mal Zahl
\end{center}
$\Rightarrow \quad Pr[A] =\frac{1}{4}, \quad Pr[B] = \frac{1}{2}$\\
$\Rightarrow \quad Pr[A \cup B] = Pr[A] + Pr[B] = \frac{3}{4}$
\item  Wir betrachten einen fairen Münze (d.h Kopf und Zahl sind gleich wahrscheinlich), welche zwei mal geworfen wird:
\begin{center}
$\Omega = \{KK,KZ,ZK,ZZ\}$\\
A:= Nur der zweite Wurf ist K\\
B:= Man wirft ein mal K\\
$A \cup B$:= Man wirft einmal K oder nur der zweite wurf ist K\\
\end{center}
$\Rightarrow \quad Pr[A] =\frac{1}{4}, \quad Pr[B] = \frac{1}{2}$\\
$\Rightarrow \quad Pr[A \cup B] = \frac{1}{2} <\frac {3}{4}$
\item Pr[A] = Pr[B] $\rightarrow$ Pr[A $\cap$ B] = Pr[A]$\cdot$ Pr[B] = $Pr[A]^2 = \frac{1}{2}$\\
$\Rightarrow \quad Pr[A] = \frac{1}{\sqrt{2}}$\\
Wir betrachten einen Münze welche mit wahrscheinlichkeit $\frac{1}{\sqrt{2}}$ Kopf und mit wahrscheinlichkeit $1-\frac{1}{\sqrt{2}}$ Zahl zeigt. Die Münze wird zweimal geworfen.
\begin{center}
$\Omega = \{KK,KZ,ZK,ZZ\}$\\
A:= Man wirft K beim ersten Wurf\\
B:= Man wirft K beim zweiten Wurf\\
\end{center}
$\Rightarrow Pr[A] = \frac{1}{\sqrt{2}}, Pr[B] = \frac{1}{\sqrt{2}}$\\
$\Rightarrow Pr[A] \cdot Pr[B] = \frac{1}{\sqrt{2}} \cdot \frac{1}{\sqrt{2}} = \frac{1}{2} = Pr[A\cap B]$
\item Ein Wahrscheinlichkeitsraum für dieses Fall existiert nicht.\\
$Pr[A \cap B \cap C] = 0$
$\Rightarrow Pr[\bar{A} \cup \bar{B} \cup \bar{C}] \leq Pr[\bar{A}] + Pr[\bar{B}] + Pr[\bar{C}] = \frac{3}{5} \ \dagger$ \\
Wobei wir die boolsche ungleichung verwendet haben. 
\end{enumerate}
\item Die Wahrscheinlichkeitsraum besteht aus den geoordneten Paare von der Wurfel zahlen und Kopt bzw Zahl. Alle ereignisse haben die gleiche wahrscheinlichkeit.\\
$\Rightarrow \quad \Omega = \{ \{1,2,3,4,5,6\} \times \{K,Z\} \}, \quad |\Omega| = 12$\\
A:= Samantha gewinnt = Samantha wurfelt ein Zahl $<$ 5 und wirft K\\
$\Rightarrow$ Samantha gewinnt wenn sie eine der ereignisse: $\{(1,K),(2,K),(3,K),(4,K)\}$ kriegt.\\
$\Rightarrow Pr[A] = \frac{4}{|\Omega|} = \frac{4}{12} = \frac{1}{3}$
\end{enumerate}
\end{document}