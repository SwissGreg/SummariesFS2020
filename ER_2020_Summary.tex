\documentclass[8pt]{extreport}
\usepackage{parskip}
\usepackage{amsmath}
\usepackage{amssymb}
\usepackage{graphicx}
\usepackage{enumitem}
\title{Entrepreneurial Risk\\ Summary}
\begin{document}
	\maketitle
	\newpage
\chapter{Introduction}
\paragraph{\underline{Types of risk:}}{
\begin{itemize}
\item \textbf{Industrial risks}
\begin{itemize}
\item change in technology, productivity, prices
\item false estimates of the rated capacity
\item time needed for the construction and running-in periods, political, social, and business enviroment
\end{itemize}
\item \textbf{Operational risks}
\begin{itemize}
\item Lack of entrepreneurship skills
\item Poor understanding of market dynamics
\item Poorly available consultancy services and information systems
\item Poor understanding of how to prepare a business plan
\item Natural risks (energy companies, constructor companies,...)
\end{itemize}
\item \textbf{Market risks}
\begin{itemize}
\item Unforeseeable inflation and exchange rates change
\item Customer behaviors to buy foreign goods
\item Inadequate infrastructure
\item Shrinking market because of foreign competitors
\item Defaulting or insolvency, Credit risks
\end{itemize}
\item \textbf{Cultural risks}

\item \textbf{Natural risks} (Hailstorm, earthquakes, hurricanes, dry seasons, and other natural hazards)

\item \textbf{Economic and political risks}
\end{itemize}

\paragraph{\underline{Formal Representation of Risk:}}
Risk is commonly measured as a pair of the probability of occurence of an event, and the outcomes or consequences associated with the event's occurrence
\begin{center}
$Risk = [(p_{1},c_{1}), (p_{2},c_{2}),\dots, (p_{n},c_{n})]$
\end{center}
\paragraph{\underline{Technical Terms:}}
\begin{itemize}
\item \textbf{Hazard}
\begin{itemize}
\item A hazard is an act or phenomenon posing potential harm to some person or thing i.e a source of harm and its potential consequences
\item Hazards need to be identified and considered in projects lifecycle analyses since they could pose threat and could lead to project failures
\end{itemize}
\item \textbf{Uncertainty}
\begin{itemize}
\item Aleatoric uncertainty: due to variability inherent in the phenomenon under consideration
\item Epistemic uncertainty: lack of knowledge, important missing mechanisms
\begin{center}
Risk = Uncertainty + Damage\\
Risk = Hazard/Safeguard
\end{center}
\end{itemize}

\item \textbf{Reliability}
\item \textbf{Event Consequences} 
\item \textbf{Performance}
\item \textbf{Risk-based Technology}
\paragraph{\underline{Risk Assessment}} consists of:
\begin{itemize}
\item Hazard identification
\item Event probability assessment
\item Consequence assessment
\end{itemize}
\paragraph{\underline{Risk Control:}} Require the definition of acceptable and comparative evaluation through monitoring and decision analysis, also includes failure prevention and consequence mitigation
\paragraph{\underline{Risk communication:}} Involves the perceptions of risk and depends on the audience targeted. Hence it is classified into:
\begin{itemize}
\item Risk communication to the media
\item To the public
\item To the engineering community
\end{itemize}

\paragraph{\underline{Human Errors:}} Human errors are unwanted circumstances caused by humans that result in deviation from expected norms that place systems at risk. It is important to identify the relevant errors to make accurate risk assessments. Human error identification techniques should provide a comprehensive structure for determining significant human errors within the system.
\paragraph{\underline{Human Error Modelling:}} Currently there is no consensus on how to model humans reliably.
\paragraph{\underline{Human Error Quantification:}} Still a developing science requiring understanding of human performance, cognitive processing, and human perceptions.

\end{itemize}
\chapter{\Large{Start-ups and Investment in Innovation}}
\paragraph{\underline{Difference between Private Equity and Venture Capital:}\\ \\}
\textbf{Private Equity} is:
\begin{itemize}
\item Financing mainly used to buy mature well-established companies and get full control (100$\%$ of company)
\item Always a combination of debt and equity (shares)
\item Value created through streamlining of operations, cost cutting, consolidation ...
\item Strong focus on cash flow to pay off debt, companies are highly leveraged (much debt relative to equity). Leverage increases risk profile but also the potential returns
\item Large deal sizes (hundreds of millions)
\end{itemize}
\textbf{Venture Capital} is:
\begin{itemize}
\item Financing mainly given to startup companies and small businesses
\item Value created through growth
\item Growth expected from innovation, disruptive technology, new product, business plan
\item Very high risk profile, no cash flow but cash burn (that is where the money is used for)
\item Company can only finance through equity, often risk profile is too high to get debt financing
\item Smaller deal sizes (millions)
\item no full control (50$\%$ or less)

\end{itemize}
\paragraph{\underline{Investors Viewpoint:}} Investors want to manage their risk with preference shares and by gaining as much control as possible on the company through shareholder agreements, voting rights, board membership, anti-dilution clauses ... The pay-off is highly skewed, there are many losers and a few big winners, hence Investors must be detached from individual companies and look at the whole portfolio. This is not aligned with the Entrepreneurs viewpoint
\paragraph{\underline{Preference Shares:}}
\begin{itemize}
\item If a company goes bankrupt, during the liquidation the investor is payed first
\item If a company is sold investors recieve payment (including compounded interest) before common shareholders (founders,...)
\end{itemize}
\paragraph{\underline{Percieved versus Actual Risk:}} Two kinds of bias were identified:
\begin{enumerate}
\item Tendency to overestimate small frequencies and underestimate larger ones
\item Tendency to exagerate the frequency of some specific causes and to underestimate the frequency of others
\end{enumerate}
Individuals assess losses and gains in an asymmetric way i.e "Losses loom larger than gains"
\begin{figure}[h!]
  \includegraphics[width=\linewidth]{viewpoint.png}
  \caption{Viewpoints of Entrepreneurs and Investors}
  \label{fig:viewpoints}
\end{figure}
\paragraph{\underline{Intrinsic:}} Focus on passion
\paragraph{\underline{Extrinsic:}} Focus on success i.e money promotion etc.
\paragraph{\underline{Challenges of exploration:}}
\begin{itemize}
\item tolerance of ambiguity
\item patience : learning-by-doing/trial and error
\item luck/serendipity
\item persistence / diligence
\item intuition
\end{itemize}
It is a high risk endeavor, where the payoff is almost impossible to estimate. Disruptive innovations are initially too small to meet the ROI-targets of large established firms. However they steadily work their way up eventually capitalizing on a crucial firstmover advantage against large, less nimble, market leaders.
\paragraph{\underline{Creativity or Rational Criticism:}} It is important to find the right balance between creativity/exploration and rationalism/exploitation. When the governance of a company is well-designed and the power between Entrepreneurs and investors is well-balanced you get the best of both worlds: The long-term perspective, personal commitment and explorative approach of the Founder, balanced with the detachment, rational, efficient, technical and financial approach of the Investor.
\begin{figure}[h!]
  \centering\includegraphics[width = 70mm, scale =1]{QuantativeRisk7.png}
  \caption{Pre/Post-Money}
  \label{scenarioList}
\end{figure}


\paragraph{\underline{NDA - Non Disclosure Agreement:}} This is a legal contract that outlines the use of confidential material, knowledge and information that the parties wish to exchange. Issues addressed in an NDA:
\begin{itemize}
\item The definition of what is confidential
\item Exclusions
\item Time period of confidentiality
\item Description of what must be done with the confidential material upon agreement ending (duty to return or destroy)
\end{itemize}
After the NDA is signed by both parties, access is given to the data room.
\paragraph{\underline{LOI - Letter of Intent:}} After a first check of the data room, the investors will produce a term sheet or LOI this,
\begin{itemize}
\item outlines an agreement that two or more parties expect to make
\item Term Sheet(TS) and LOI are very similar in content but TS is structured as a list (table format), whereas LOI is in the form of a letter
\item Written before the execution of a formal and binding contract, most of the listed agreements are not legally binding
\end{itemize}
\paragraph{\underline{Exclusivity:}}
\begin{itemize}
\item Legally binding clause of the LOI or TS
\item Caveat: Transfers a lot of control to the investor, they will be the only party taking the next step in the process and can take advantage of the 'sunk cost effect'. When given exclusivity, time is on the side of the investor
\item The investor can put you under pressure, test your tenacity and patience, try to decrease the valuation of the company. When you give exclusivity you cancel out any competition for the investor, this will make them dominant.
\end{itemize}
\paragraph{\underline{Due Diligence:}} After the Term Sheet and/or LOI are signed the Due Diligence process is started. Now external advisors enter the arena:
\begin{itemize}
\item Business lawyers review all the contracts in the data room
\item IP lawyers will study the strengths of the patents of the company and the 'freedom to operate' (with respect to the patents of other companies)
\item External Auditor will validate the accounting, financial statements, balance sheet, taxes
\item Tech-consultant may analyze the product development and the strength and relevance of the tech with respect to other solutions
\end{itemize}
The investors themselves will:
\begin{itemize}
\item Reference check clients, founders, key personnel
\item Analyze the commercial viability of the product and the sales process and tools
\item Study the quality of the sales pipeline
\item make their own forecast of future sales and of future cashflows
\end{itemize}
Based on the results of the due diligence, the investors will challenge the business plan:
\begin{itemize}
\item Create base-case and worst-case scenarios of cash burn
\item Confirm if the founders are asking for too much or too little
\item Assess the risk of their investment
\item Set management goals
\item Find the gaps i.e where the company needs further support and make their own valuation of the company
\end{itemize}
\chapter{\Large{Introduction to company valuation}}
\paragraph{\underline{Reports for a company:}}
\begin{itemize}
\item balance sheet
\item profit and loss acount 
\item cash flow statement
\end{itemize}

\paragraph{\underline{Enterprise Value vs. Equity:}}
\begin{itemize}
\item \textbf{Enterprise Value} is the price to acquire the whole company, the shares, debt but also receiving the cash.
\item \textbf{Equity Value} is the price to acquire only the shares of a company
\item \underline{ EnterpriseValue(EV) = Equity + Debt - Cash}
\end{itemize}

\paragraph{\underline{LLC in Switzerland}} To set up a LLC in Switzerland we need to pay 20k CHF out of pocket
\paragraph{\underline{ASSETS = LIABILITIES + EQUITY}\\ \\} Looking at equity in a balance sheet is one way to assess the value of the company. At the beginning the company is only worth the amount that we have put into it (nothing has happened yet).
\paragraph{\underline{Balance Sheet:}} Table keeping track of Assets, Liabilities and equity. The balance sheet is dynamic i.e it changes every year. Each year, depreciation of assets, equity increases by the profit amount, liabilities decrease if paid off.
\paragraph{\underline{Return on Equity:}} A measure of profibility of the company. The gain over the past year.
\begin{center}
 $ROE = \frac{EQUITY}{PROFIT}$
\end{center}
\paragraph{\underline{Leverage:}}  Measures how much money you put in the company yourself to how much money you attract. Leverage increases your risk proportionally. The larger the business is, the higher the losses can be. Increasing the business with debt, without increasing the equity buffer proportionally, you will have a higher ROE but also a much higher risk of insolvency (not being able to service the larger amount of debt). When continuing to accumulate losses, at some time your full equity buff will be consumed. Because of the cash you burn, debts become higher than your assets, your company has "negative equity" this is a sign of future insolvency. Insolvency and negative equity may lead to debt restructuring or bankruptcy. You need the right balance between debt and equity (1/1, 3/2  max 2/1 debt/equity).
 \begin{center}
$LEVERAGE=\frac{LIABILITIES}{EQUITY}$
\end{center}
\paragraph{\underline{Ways to evaluate a company:}}
\begin{itemize}
\item A company's value can be deduced from its balance sheet. This is a very static approach. It takes a snapshot of the equity in the balance sheet and does not consider how this was acquired (one year, ten years?) A company is a "process" that generates profit (or loss) based on people, strategy and stuff(assets,tech,...). Hence it makes sense to use metrics from the P$\&$L to value the company.
\end{itemize}
\paragraph{\underline{Banking:}} E.g Taxi business: When starting the business, if you make losses your equity buffer goes down, if you make profits the buffer will increase. For banks, assets (bonds, stocks, loans(largest asset),...) may lose value due to systemic market events,and equity may decrease due to accumulated losses. Banks operate using large leverage ratios (In 2007 40/1 hence a drop of their assets by 2.5$\%$ would leave them with a negative equity). The government calculates how much equity a bank must have. The banking system is closely knitted hence if Bank A collapses then bank B will also get in trouble and they cannot provide loans to companies. Resulting in the whole economic system collapsing. Too much leverage in the system may make it very vulnerable and susceptible to systemic shocks.
\paragraph{\underline{Profit and Loss:}} 
\begin{itemize}
\item \textbf{Revenue}(sales): is the top line i.e how much money is generated by the business per year
\item \textbf{Costs of goods sold} (COGS): costs of material and labour to run the business
\item \textbf{Gross Margin}: 
\begin{center}
\underline{Gross Margin = Revenue - COGS}
\end{center}
\item \textbf{Operating Expenses} (OPEX): Remaining costs that are not included in COGS (costs of office and equipment and other overhead  e.g depreciation of taxis in taxi business)
\item \textbf{ Earnings Before Interest and Taxes} (EBIT): Operating profit of the company
\begin{center}
 \underline{EBIT = Gross Margin - OPEX}
\end {center}
 (6$\%$ is good for a small company. It is the part of sales that belong to the owner and debt owner)
\item \textbf{Earnings Before Interest,Taxes, Depreciation and Amortization} (EBITDA): 
\begin{center}
\underline{EBITDA = EBIT + D$\&$A}
\end{center}
\item \textbf{Earnings Before Taxes} (EBT):
\begin{center}
\underline{ EBT = EBIT - Interests paid}
\end{center} 
\item \textbf{Net Profit}: part of sales that belong to the owner.
\begin{center}
\underline{Net Profit = EBIT - Interests paid - taxes}
\end{center}
\item \textbf{Margin Sensitivity}: How sensitive is the Net profit to fluctuations in the cost
\end{itemize}
\paragraph{\underline{Relative Valuation:}} Do not compare companies based on the stock price. Look at companies in the same field and compare their performance in the market. Find a normalized share price. An important metric is $\frac{price}{earnings}$ i.e price of the company divided by the earnings per year. Public companies are more expensive than private companies because private companies are less liquid and hence cannot be sold as easily (Liquidity premium)
\paragraph{\underline{Valuation Multiples:}}
\begin{itemize}
\item \textbf{P/E} ratio: Calculate a company's share price (Equity) from its earnings (net profit)
\item \textbf{EV/EBITDA} ratio: Calculate a company's Enterprise Value from its EBITDA
\item \textbf{EV/EBIT} ratio: Calculate a company's Enterprise Value from its EBIT
\end{itemize}
\paragraph{\underline{Components of Cash Flow:}}
\begin{itemize}
\item \textbf{Operating Activities:} e.g Income - living costs
\item \textbf{Investing Activities:} e.g Car, House 
\item \textbf{Financing Activities:}  e.g Bank Loan to buy Car, House
\end{itemize}
\subparagraph{\underline{Operating Activities Cash Flow:}}
\begin{center}
$Operating Cash Flow = Net Income + Non Cash Expenses - Increase In Working Capital$
\end{center}
\begin{itemize}
\item $\delta$ WC = Change in Accounts Receivable + Change in Inventory - Change in Accounts payable
\item \textbf{Accounts receivable} = Sum of all invoices sent out to customers that have not paid yet
\item \textbf{Accounts payable} = Sum of all invoices recieved from vendors that you have not paid yet
\end{itemize}
\subparagraph{\underline{Investing Activities Cash Flow:}}
\begin{center}
$Investment Activities Cash Flow = (Purchase/Sale) \  Long Term Assets + (Purchase/Sale) Businesses + (Purchase/Sale) Marketable Securities$
\end{center}
\subparagraph{\underline{Financing Activities Cash Flow:}}
\begin{center}
$Financing Activities Cash Flow = (Issue/Repurchase) Equity + (Issue/Repurchase) Debt + Dividend Payments$\&$Other Items$
\end{center}
\paragraph{\underline{Risk of fast growth:}} As a startup when you grow very fast, your working capital can also increase very quickly because:
\begin{itemize}
\item You have to pay vendors early (in advance) because they do not trust you (being a young company) and otherwise they will not supply
\item You do not get paid by clients because they have strong negotiating power
\item You have to increase your inventory
\end{itemize}
$\Rightarrow$ You can get in serious liquidity problems and even go bankrupt because you grow too fast

\paragraph{\underline{Time value of money:}}
\begin{itemize}
\item FV = Future Value
\item PV = present Value
\item n = number of periods in years
\item r = rate of return or discount rate or interest rate or growth per period
\end{itemize}
\begin{center}
$ FV = PV \cdot (1 + r)^n$
\end{center}
\paragraph{\underline{Bond}:} A bond is a loan that you can trade in the market i.e a liquid loan. A loan pays you a coupon which pays you a percentage of the value you bought it at:
\begin{center}
$ Bond Price = \frac{C}{1+i} + \frac{C}{(1 + i)^2} + \dots +\frac{C}{(1+i)^n} + \frac{M}{(1+i)^n}$
\end{center}
\begin{itemize}
\item C = coupon payment
\item n = number of payments
\item i = interest rate, or required yield
\item M = value at maturity or par value
\end{itemize}
$\Rightarrow$ The same principle can be applied to value a company, this is called Discounted Cash Flow Valuation
\paragraph{\underline{Discounted Cash Flow Valuation:}}
\begin{center}
$PV = \frac{CF_{1}}{(1+r)^1} + \frac{CF_{2}}{(1+r)^2} + \frac{CF_{3}}{(1+r)^3} \dots \frac{CF_{n}}{(1+r)^n}$
\end{center}
\begin{itemize}
\item CF = cash flow for a period
\item r = discount rate
\item n = number of periods
\item The value of a company can be calculated as the present value of a string of cash flows
\item There is no maturity so n = $\infty$
\item Cash flows are not simple coupons like with a bond but must be estimated
\item r is not the yield of the bond, but is a discount rate that reflects the return that the investor expects under the base case. It depends on the risk perception of the investor, high risk will be high r and vica versa
\end{itemize}
If the company grows steadily at a rate g then the PV can be calculated as follows:
\begin{center}
$PV = \frac{CF_{1}}{r-g}$
\end{center}
Whereas Valuing a young company with a 5 year start-up phase:\\
\begin{center}
$PV = \displaystyle\sum_{i=1}^{5}\frac{CF_{i}}{(1+r)^i} + \frac{TV}{(1+r)^5} = \displaystyle\sum_{i=1}^{5}\frac{CF_{i}}{(1+r)^i} + \frac{\frac{CF_{6}}{r' - g}}{(1+r)^5}$
\end{center}
\begin{itemize}
\item r' = target rate of return of mature business
\item r = target rate of return during startup phase
\item We discount the estimated cash flows during the startup phase individually
\item We calculate the TV(Terminal Value) of the company 5 years in the future (i.e the 6th CF), when it has reached a state of maturity growth
\item We discount this TV
\end{itemize}
\section{Bubbles}
\paragraph{\underline{Normal Stock behavior:}} Share prices follow a Geometric Brownian Motion (Random walk motion) composed of a drift (expected price increase of that stock) and a noise i.e the risk part of the equation ( the volatility of the stock). The Geometric Brownian Motion random walk implies that prices follow an exponential track decorated with noise. In real markets, growth rates are not stable.
\paragraph{\underline{Regime:}} A time period where stock prices dont follow brownian Motion and change faster than exponential. (Trick to detect bubbles: if you switch to a log scale and increase is still high)
\paragraph{Efficient Market Hypothesis}
\paragraph{\underline{Description of Bubble:}}
\begin{itemize}
\item Bubble starts with a new opportunity or expectation
\item Smart money flows in, which leads to a first price appreciation
\item Attracted by the prospect of higher returns, less sophisticated investors follow
\item Demand goes up as the price increases and the price goes up as the demand increases. This creates a positive feedback mechanism. The market is fully driven by behavior and sentiment and no longer reflects any real underlying value
\item At some point investors start realizing that the process is no longer sustainable and the market collapses
\item The crash occurs because the market has entered an unstable phase.
\item The mechanism is often not well understood and a great controversy rises about the cause of the crash
\end{itemize}
\paragraph{\underline{exogenous process:}} Cause and effect are linearly and logically connected (What is important is the trigger e.g asteroid hitting earth, state of earth is irrelevant)
\paragraph{\underline{endogenous processes:}} cause and effect are not linearly connected (What is important is the state of the system, any small event can trigger a major incident)
\paragraph{\underline{Complex Systems:}}
\begin{itemize}
\item Consist of a large ensemble of agents e.g molecules, stars, animals, humans etc
\item These interact e.g repel, attract, imitate etc.tem There is Emergence i.e local interactions lead to global cooperation in absence of any global orchestrations
\end{itemize}
\chapter{\Large{Wrapping up the deal}}
\paragraph{\underline{Before drafting the legal documents:}} There must pe an agreement on the most important principles of the deal:
\begin{itemize}
\item \textbf{Amount Invested} and how it will be made available (single paymen or tranches)
\item \textbf{Value of the company} (with distinction of pre- and post money valuation)
\item  \textbf{Cap table} i.e who owns what percentage of the company
\item  \textbf{Governance principles}
\end{itemize}
The order is important!
\paragraph{\underline{Amount Invested:}} \small{Based on all the information gethered during the due diligence the investor will calculate a number of cash burn scenarios. This will be used to assess the amount of capital that needs to be invested.
\textbf{Capital invested in a startup is used for burning cash, so its important to have a good understanding of different cash burn scenarios.} Investors may have an incentive to start with a low initial investment, if things go well they will have a higher ROI, if things go bad, they can invest the additional amount at a lower company valuation. Hence it is important as a founder/entrepreneur to negotiate for a strong cash buffer. If the cash burn is higher than expected, you may need to find new capital in a situation under stress. At that time your company valuation will be low and you will dilute (i.e losing ownership of the company)}
\newpage
\paragraph{\underline{Value of the company:}} Example Below: They wanted to scale up Revenues by a factor of 10 in the next 3 years (didn't happen hence they needed a second round of investments). Steps of Analysis:
\begin{enumerate}[label=(\roman*)]
\item Calculate Gross Margin by Revenues - COGS = 777 - 358 = 419 
\item Subtract OPEX from the Gross Margin to get EBIT i.e 419 - 459 = -40
\item EBT = EBIT - interests = -40 - 9 = -49
\item Calculate the Cash Flows
\item You notice that there is a big cash out, this comes from the big OPEX(hiring for sales) and CAPEX(investments) i.e big fixed costs but the return only comes a year later hence the first years will be negative, hence to valuate the company you must discount these cash flows back aswell as its Terminal Value when the company has reached a constant growth rate.
\item the Terminal Value can be calculated using a multiple (here the EV/EBIT multiple which was 10.0)
\item with the given ROE we can discount the TV and calculate the Discounted Cash Flow(DCF) = EV
\item Pre Money is calculated by adding the Cash and subtracting the Debt from the DCF
\end{enumerate}

\begin{figure}[h!]
  \centering\includegraphics[width = \linewidth, scale =1]{QuantativeRisk4.png}
  \caption{Calculation of Pre-Money valuation}
  \label{scenarioList}
\end{figure}\newpage
\paragraph{\underline{Pre - and post money valuation and cap table:}}
\begin{figure}[h!]
  \centering\includegraphics[width = \linewidth, scale =1]{QuantativeRisk5.png}
  \caption{Pre/Post-Money}
  \label{scenarioList}
\end{figure}
\small{The Pre Cap Table is the situation before the investor puts money into the company.}\\
\begin{center}
$Dilution = \frac{(Pre \ Money \ Ownership) \cdot (Total \ value \ of \ company \ pre \ money)}{Total \ value \ of \  company \ post \ money}$
\end{center}
\paragraph{\underline{Governance Principles:}}
\small{
\begin{itemize}
\item \underline{\textbf{What?}}\\ To outline the responsibility, composition and authority(decision making process) of the Management Team(MT), Supervisory Board(SVB) and General Meeting of Shareholders(GMS) of the company. (These are the 3 Corporate bodies of a company)
\item \underline{\textbf{Why?}}\\ To allow for an efficient management of the company based on objective criteria and processes independently of existing persons and historical relationships
\item \underline{\textbf{How?}}\\ By writing or changing the articles of association and/or the shareholder agreement, where needed
\end{itemize}
}
\paragraph{\underline{Corporate bodies:}\\ \\}

\underline {Management Team:} \textbf{Day-to-day affairs}
\begin{itemize}
\item Directs the company's day to day affairs
\item Has the authority to decide in line with the annual budget and business plan that has been approved by the board
\item In case of a significant deviation from the budget and the business plan, the decision is escalated to the board
\item Often a matrix is drafted showing clearly the decision authority of each of multiple MT members ordered according to subject, amount, signing authority...
\end{itemize}
\underline {Supervisory Board:} \textbf{Supervision}
\begin{itemize}
\item Composed of representatives of the shareholders + independent board members(non-executive) + senior management (executive)
\item Composition and voting rights are clearly defined in the shareholders agreement
\item Must supervise and advise the management and oversee the general affairs within the company
\item Should be guided by the interests of the company
\end{itemize}
\underline{Typical decision authority of the board:}
\begin{itemize}
\item Hire/fire of senior management
\item Adoption and/or amendment of yearly business plan and budget
\item Investments, loans, contracts ... exceeding a threshold
\item Option plan for employees
\item Targets and variable remuneration of senior management
\end{itemize}
\underline{General meeting of shareholders:} \textbf{Value Creation}
\begin{itemize}
\item Composed of the shareholders (owners) of the company
\item May give priority to their own interests with due regard for the principles of reasonableness and fairness
\item Meets at least once per year to approve the annual accounts, discharge the board and follow up and/or adapt the Value Creation Plan (Long term business plan)
\item Appoints the members of the Supervisory Board and sometimes also members of the management team (like the CEO)
\item Decides by majority unless explicitly stated differently in the shareholder agreement
\end{itemize}
\underline{Typical decision authority of shareholders:}
\begin{itemize}
\item issue new shares
\item hire/fire new CEO
\item distribution of dividends
\item Reorganisation of the business
\item Application for bankruptcy
\item $\dots$
\end{itemize}
$\Rightarrow$ Any decision which will largely influence the value of shares
\paragraph{\underline{Legal Documents:}}
\begin{itemize}
\item Subscription Agreement (the transaction)
\item Shareholders Agreement (governance and organisation)
\item Management Agreement (day-to-day operations)
 \end{itemize}
\underline{Subscription Agreement} \textbf{The Transaction}
\begin{itemize}
\item A Subscription Agreement is between a company and a private investor to sell a specific number of shares ata specific price.
\item it contains amongst others, information regarding the amount invested, the cap table, issue of new shares or transfer of existing shares, payment conditions, conclusions on the due diligence, warranties, $\dots$
\item Some agreements include a specified rate of return that investors are guaranteed to receive ("preference Shares")
\end{itemize}
\underline{Shareholders Agreement} \textbf{Governance and Organisation}
\begin{itemize}
\item A shareholders agreement describes how the company should be operated and outlines shareholders rights and obligations.
\item Is intended to make sure that all shareholders are treated fairly and that their rights are protected
\item Outlines the governance principles: the responsibility, the composition and the authority of the MT, SVB and the GMS of the company
\item Describes the exit scenarios (transfer of shares) with specific care for the rights of minority as well as majority shareholders
\end{itemize}
\paragraph{\underline{Lock-up:}} A predetermined amount of time where shareholders are restricted from selling their shares
\paragraph{\underline{Right of first refusal:}} After the lock up period, when one shareholder can sell shares to a third party, the other shareholders must be given the opportunity to match the price and buy shares instead of the third party
\paragraph{\underline{Drag along:}} (Protection of majority shareholder) A drag along right allows a majority shareholder of a company to force the remaining minority shareholders to accept an offer from a third party to purchase the whole company at the same price,terms and conditions. Drag-along rights help eliminate minority owners and sell 100$\%$ of a company's securities to a potential buyer.
\paragraph{\underline{Tag along:}} (protection of minority shareholder) Tag along rights are the inverse of drag along rights. When a majority shareholder sells their shares, a tag along right will entitle the minority shareholder to participate in the sale at the same time for the same price, terms and conditions. The minority shareholder then tags along with the majority shareholder's sale.
\paragraph{\underline{Good/Bad leaver clause:}} A description of the circumstances in which a person ceases to be an employee of a company (For founders, often this leads to forced selling of the shares)
\paragraph{\underline{Good leave:}} Usually due to illness,disability to work, death or (early) retirement. (Founders get market value for their shares)
\paragraph{\underline{Bad leave:}} Voluntary leave before end of contract, compelling cause e,g criminal activity (Founders get way less than market value for their shares)
\paragraph{\underline{Negotiations:}} Finalisation of the legal documents may take quite some time. There will be negotiations and small print will be read and discussed in great detail.
\paragraph{\underline{Final Step:}} Signing at the notary office
\begin{figure}[h!]
  \centering\includegraphics[width = \linewidth, scale = 0.5]{QuantativeRisk6.png}
  \caption{}
  \label{scenarioList}
\end{figure}
\chapter{\Large{On The Quantative Definition of Risk:}\\ \normalsize{(Paper by Stanley Kaplan, John Garrick)}}
\section{\large{Qualitative aspects of the notion of risk:}}
\paragraph{Distinction Between Risk and Uncertainty:} The notion of risk involves both uncertainty and some kind of loss or damage that might be received, hence
\begin{center}
$ risk = uncertainty + damage$
 \end{center}
\paragraph{Distinction Between Risk and Hazard:} Hazard exists as a source, risk includes the likelihood of conversion of that source into actual delivery of loss, injury or some form of damage. Safeguard is something to protect us against the hazard and minimize the risk. It also includes the idea of awareness of risk.
\begin{center}
$\frac{hazard}{safeguards}$
\end{center}
\textbf{Note:} The Risk can become small, but it is never zero
\paragraph{Relativity of Risk:} Risk is relative to the observer and is a subjecive thing (e.g Rattlesnake in Mailbox)
\section{\large{Quantative definition of risk:}}
\paragraph{Set of Triplets idea:} A risk analysis consists of an answer to the following questions:

\begin{enumerate}[label = (\roman*)]
\item What can happen? (What can go wrong)
\item How likely is it that that will happen?
\item If it does happen, what are the consequences?
\end{enumerate}
If we can give each possible scenario a probability and the respective damage, we have found the risk.
\begin{center}
$R = \{ \langle s_i, p_i, x_i \rangle \}, \quad i = 1,2, \dots , N$
\end{center}

\begin{figure}[h!]
  \centering\includegraphics[width = 60mm, scale = 0.5]{QuantativeRisk1.png}
  \caption{}
  \label{scenarioList}
\end{figure}
\paragraph{\underline{Risk Curves:}} \small{We now sort the scenarios by increasing severity of damage and add a column containing the cumulative probability. Plotting this results in a stair case function, which can be seen as a discrete approximation to a continuous reality. Hence we can construct a smoothed curve R(x) representing the actual risk i.e the "risk curve".}
\begin{figure}[h!]
  \centering\includegraphics[width = 60mm, scale = 0.5]{QuantativeRisk2.png}
  \caption{\footnotesize{Risk curve}}
  \label{scenarioList}
\end{figure}
\paragraph{Multidimensional Damage} \small{In many applications, it is appropriate to identify different types of damage, hence the damage x can be regarded as a vector quantity rather than a scalar. The risk curve becomes a risk surface.
\section{\large{Probability}}
\paragraph{The Definition of Probability and Distinction Between Probability and Frequency}
\begin{itemize}
\item \underline{ \textbf{probability:}} A numerical measure of a state of knowledge, a degree of belief, state of confidence
\item \underline{ \textbf{frequency:}} The outcome of an experiment involving repeated trials.
\end{itemize}
$\Rightarrow$ \small{In concept, frequency is a well defined, objective, measurable number. Probability on the otherhand is changeable and subjective.}
\paragraph{Distinction between Probability and Statistics:} \small{Statistics is the study of frequency type information i.e the science of handling data, wheras Probability is the science of handling the lack of data.}
\section{\large{Level 2 Definition of Risk}}
\paragraph{Risk Curves in Frequency Format:} \small{We adjust the triplets and execute a thought experiment, keeping track of how many times each scenario occured. We then compute the cumulative frequency $\phi_i = \displaystyle\sum_{x_j \geq x_i} \phi_i$ and plot it against x ( sum is over all scenarios having damage equal to or greater than $x_i$}
\paragraph{Inclusion of Uncertainty:} \small{When creating the risk curve we have not yet actually done the experiment  hence we have uncertainty of what its outcome would be, which is proportional to the total state of knowledge as of right now. Since the thing we are uncertain about is a curve $\phi(x)$, we express the uncertainty by embedding this curve in a space of curves and creating a probability distribution over this space.}
\begin{figure}[h!]
  \centering\includegraphics[width = 60mm, scale = 0.5]{QuantativeRisk3.png}
  \caption{\footnotesize{Risk curve in probability of frequency format}}
  \label{scenarioList}
\end{figure}
\paragraph{Set of Triplets Including Uncertainty} \small{When the frequency with which scenario category $s_i$ occurs we can express our state of knowledge about this frequency with a probability curve $p_i(\phi_i)$, which is the probability density function for the frequency $\phi_i$, of the ith scenerio.}
\begin{center}
$R = \{ \langle s_i, p_i(\phi_i), x_i \rangle \}$
\end{center}
\small{This set of triplets is the risk including uncertainty in frequency}
\section{\large{"Acceptable Risk"}}
\paragraph{Difficulties with the notion of acceptable risk:}
\begin{enumerate}[label =(\roman*)]
\item The minor difficulty is that it implies that risk is linearly comparable i.e it implies that one can say that risk of course of action A is greater or less than that of design B, which they are not. It would be possible to reduce the risk curves to a single number, but alot of information would be lost.
\item The major difficulty is that risk cannot be spoken of as acceptable or not in isolation, but only in combination with the costs and benefits that are attendant ot that risk. One must adobt a decision theory point of view and ask: "What are my options, what are the costs, benefits and risks of each?"
\end{enumerate}
\end{document}
