\documentclass[8pt]{extreport}
\usepackage{parskip}
\usepackage{amsmath}
\usepackage{amssymb}
\usepackage{fancyhdr}
\usepackage{enumitem}
\setlength{\headheight}{15.2pt}
\pagestyle{fancy}
\fancyhead[L]{A$\&$W T6}
\fancyhead[R]{Matteo Nussbaumer, Gregory Rozanski}
\title{Analysis Serie 4}
\begin{document}
\paragraph{\underline{Exercise 1 - Bälle und Körbe}}
\begin{enumerate}[label = (\alph*)]
\item Wir definieren unser Ereignisraum:\\
$\Rightarrow$ Für ein Wurf definieren wir den ereignisraum $\Omega = \{1,2...n\}$\\
$\Rightarrow$ Für m Würfe definieren wir den ereignisraum $\Omega' = \{1,2,...,n\}^m \rightarrow |\Omega| = n^m$\\
$\Rightarrow$ Gegeben der ereignis $\omega = \omega_1\dots \omega_m$ (wobei die $\omega_i$'s zueinander unabhängig sind) die Wahrscheinlichkeit dass wir den j-ten Korb beim i-ten wurf treffen:\\ $Pr[\omega_i = j] = \frac{1}{n}$
\item $A_i:=$ "nach dem i-ten wurf ist der erste Korb leer.\\
$\Rightarrow Pr[A_1] = \frac{n-1}{n}$\\
$\Rightarrow Pr[A_i] = Pr[A_{i-1}] Pr[A_1] = \big(\frac{n-1}{n}\big)^i$\\
$\Rightarrow Pr[A_m] = \big(\frac{n-1}{n}\big)^m$
\item X = $\#$ Körbe, die leer sind nachdem der Prozess geendet hat.\\
$\Rightarrow X_i:=$ Indikatorvariable dass der i-ten Korb ist leer nachdem der Prozess geendet hat\\
$\Rightarrow X = X_1 + X_2 + \dots + X_n$\\
$\Rightarrow X_i = \big(\frac{n-1}{n}\big)^m$\\
$\Rightarrow \mathbb{E}[X] = \mathbb{E}[X_1] + \mathbb{E}[X_2] + \dots \mathbb{E}[X_n]$
$\Rightarrow \mathbb{E}[X] = n \cdot \big(\frac{n-1}{n}\big)^m$
\item Y := $\#$ Bälle, die im ersten Korb landen\\
$\Rightarrow Y_i = $ Indikatorvariable dass der i-te Ball im ersten Korb landet\\
$\Rightarrow Y = Y_1 + Y_2 + \dots + Y_m$\\
$\Rightarrow \mathbb{E}[Y] = \mathbb{E}[Y_1] + \mathbb{E}[Y_2] + \dots + \mathbb{E}[Y_m]$\\
$\Rightarrow \mathbb{E}[Y_i] =\frac{1}{n}$\\
$\Rightarrow \mathbb{E}[Y] = \frac{m}{n}$\\
\item A:= Wir treffen ein Korb welche nicht die ersten beide sind\\ B:= Die ersten zwei Körbe sind leer nach alle Würfe\\
$\Rightarrow Pr[A] = \frac{n-2}{n}$ \\
$\Rightarrow$ m würfe $\rightarrow Pr[B] = \big(\frac{n-2}{n}\big)^m$
\end{enumerate}


\end{document}