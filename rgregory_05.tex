\documentclass[8pt]{extreport}
\usepackage{parskip}
\usepackage{amsmath}
\usepackage{amssymb}
\usepackage{fancyhdr}
\usepackage{graphicx}
\usepackage{enumitem}
\usepackage{mathtools}
\setlength{\headheight}{15.2pt}
\pagestyle{fancy}
\fancyhead[L]{Analysis Serie 5}
\fancyhead[R]{Gregory Rozanski}
\title{Analysis Serie 5}
\begin{document}
\paragraph{\underline{ MC Fragen: Folgenkonvergenz}}
\begin{enumerate}[label = (\alph*)]
\item $\lim\limits_{n \to \infty} sup |a_n|^{\frac{1}{n}} = \lim\limits_{n \to \infty} sup |b_n|^{\frac{1}{n}}$
\item
\begin{enumerate}[label =(\Alph*)]
\item konvergiert immer, aber nicht unbedingt absolut
\item konvergiert nicht umbedingt
\end{enumerate}
\item konvergiert nicht umbedingt
\end{enumerate}
\paragraph{\underline{5.2 Reihen reellen Zahlen: (Bitte Korrigieren)}}
\begin{enumerate}[label = (\alph*)]
\item Wir verwenden den QK: $a_n = \frac{{n!}^2}{(2n)!} \quad  a_{n+1} = \frac{(n+1)!^2}{(2n+2)!}$\\
$\Rightarrow \lim\limits_{n \to \infty} \Bigg| \frac{ \frac{(n+1)!^2}{(2n+2)!}}{ \frac{{n!}^2}{(2n)!}}\Bigg| \Rightarrow \lim\limits_{n \to \infty}\bigg|\frac{(n+1)^2}{(2n+1)(2n+1)}\bigg|  \Rightarrow \lim\limits_{n \to \infty}\bigg|\frac{n^2 + 2n + 1}{4n^2 + 6n + 2}\bigg| = \bigg|\frac{n^2}{4n^2}\bigg| = \frac{1}{4} = q$\\ $\Rightarrow$ die Reihe \underline{Konvergiert} absolut
\item $\displaystyle\sum_{n = 1}^{\infty} \frac{1}{n+100} = \displaystyle\sum_{k=1}^{\infty}\frac{1}{k} - \displaystyle\sum_{k=1}^{100}\frac{1}{k} \Rightarrow $Harmonische Reihe divergiert.$\displaystyle\sum_{n = 1}^{\infty} \frac{1}{n+100}$ \underline{Divergiert}
\item Klare Nullfolge. Wir verwenden den WK. $|a_n|^\frac{1}{n} = \bigg|\frac{5^n}{n^{n+1}}\bigg|^\frac{1}{n} = \frac{5}{n^{1+\frac{1}{n}}} \xrightarrow{n \to \infty} 0$\\
$\Rightarrow$ Die Reihe \underline{Konvergiert Absolut}.
\item $|a_n| = |-1|^n\bigg|\frac{n+1}{2n+1}\bigg| = \bigg|\frac{n+1}{2n+1} \bigg| \xrightarrow{n \to \infty} \frac{1}{2} \neq 0 \xrightarrow{NFK}$ die Reihe \underline{Divergiert}.
\item Klare Nullfolge. $|a_n| = \frac{1}{n(n+4)} \leq \frac{1}{n^2} = \zeta(2) = b_n \Rightarrow \displaystyle\sum_{n=1}^{\infty}\frac{1}{n(n+4)}$ \underline{Konvergiert Absolut}
\end{enumerate}
\paragraph{\underline{5.3 Reihe I}}
\begin{enumerate}[label =(\alph*)]
\item $\lim\limits_{n \to \infty} \bigg| \frac{1 + \alpha^n}{1 + \beta^n} \bigg|^\frac{1}{n} = \lim\limits_{n \to \infty} \frac{\alpha(\frac{1}{\alpha^n} + 1)^\frac{1}{n}}{\beta(\frac{1}{\beta^n}+1)^\frac{1}{n}} = \frac{\alpha}{\beta} \Rightarrow \rho = \frac{\beta}{\alpha}$
\item $|a_n|^\frac{1}{n} = \frac{|(-3)^n|^\frac{1}{n}}{|n+2|^\frac{1}{n}} = \frac{3}{(n+2)^\frac{1}{n}} \xrightarrow{n \to \infty}  3 \Rightarrow \rho = \frac{1}{3}$\\
$\Rightarrow$ für $|x| \leq \frac{1}{3}$ konvergiert die Potenzreihe.
\end{enumerate}
\end{document}